% ****** Start of file apssamp.tex ******
%
%   This file is part of the APS files in the REVTeX 4.1 distribution.
%   Version 4.1r of REVTeX, August 2010
%
%   Copyright (c) 2009, 2010 The American Physical Society.
%
%   See the REVTeX 4 README file for restrictions and more information.
%
% TeX'ing this file requires that you have AMS-LaTeX 2.0 installed
% as well as the rest of the prerequisites for REVTeX 4.1
%
% See the REVTeX 4 README file
% It also requires running BibTeX. The commands are as follows:
%
%  1)  latex apssamp.tex
%  2)  bibtex apssamp
%  3)  latex apssamp.tex
%  4)  latex apssamp.tex
%
\documentclass[%
%%%%%ok reprint,
%superscriptaddress,
%groupedaddress,
%unsortedaddress,
%runinaddress,
%frontmatterverbose, 
preprint,
%showpacs,preprintnumbers,
nofootinbib,
%nobibnotes,
%bibnotes,
 amsmath,amssymb,
 aps,
%pra,
%prb,
%rmp,
%prstab,
%prstper,
floatfix,
]{revtex4-1}

\usepackage{graphicx}% Include figure files
\usepackage{setspace}
\usepackage{dcolumn}% Align table columns on decimal point
\usepackage{bm}% bold math
%\usepackage[caption=false]{subfig}% bold math
\usepackage{subcaption}% bold math
%\usepackage{hyperref}% add hypertext capabilities
%\usepackage[mathlines]{lineno}% Enable numbering of text and display math
%\linenumbers\relax % Commence numbering lines

%\usepackage[showframe,%Uncomment any one of the following lines to test 
%%scale=0.7, marginratio={1:1, 2:3}, ignoreall,% default settings
%%text={7in,10in},centering,
%%margin=1.5in,
%%total={6.5in,8.75in}, top=1.2in, left=0.9in, includefoot,
%%height=10in,a5paper,hmargin={3cm,0.8in},
%]{geometry}

\begin{document}

\preprint{Preliminary notes}

\title{ 
Technical requirements for the GlueX TRD gas system 
}% Force line breaks with \\
%\thanks{Thanks to everybody who contributed to this work}%
\author{Sergey Furletov}
\author{Lubomir Pentchev}
\affiliation{Thomas Jefferson National Accelerator Facility, Newport News, Virginia 23606, USA}
%
%\author{Ann Author}
% \altaffiliation[Also at ]{Physics Department, XYZ University.}%Lines break automatically or can be forced with \\
%\author{Second Author}%
% \email{Second.Author@institution.edu}
%\affiliation{%
% Authors' institution and/or address\\
% This line break forced with \textbackslash\textbackslash
%}%

%\collaboration{GlueX Collaboration}%\noaffiliation


\date{\today}% It is always \today, today,
             %  but any date may be explicitly specified

\begin{abstract}
Short description of the detector is given and preliminary requirements for the TRD gas system are specified. 
\end{abstract}

%\pacs{Valid PACS appear here}% PACS, the Physics and Astronomy
%                             % Classification Scheme.
%%\keywords{Suggested keywords}%Use showkeys class option if keyword
%                              %display desired
\maketitle

%\tableofcontents

\clearpage
\mbox{~}
%\section{Formulation of the problem}

The future GlueX TRD detector will consist of three modules.
Each module is a box of $1700$x$1700$~mm$^2$ cross section, with two gas volumes - 
the main one filled with $Xe/CO_2$ gas mixture of $90/10\%$ 
containing the drift and the amplification volumes,
and the second one for the radiator filled with CO2. 
The thickness of the drift and amplification volume is $25$~mm and $10$~mm respectively.
Thus, we estimate the $Xe/CO_2$ gas volume to be $100$~l per module, or $300$~l in total.
For the $CO_2$ volume it has a thickness of $150$~mm or $430$~l per module and $1300$~l total.
For the $Xe/CO_2$ volume we aim to have 8 volume exchanges per day, i.e. $100$~l/h.
The $CO_2$ volume can be exchanged once  per day or $55$~l/h.

The entrance and exit gas windows will be made of $100$~$\mu$m Mylar, possibly enforced with Rochacell
material.
The detector will allow operation with overpressure between $0.5$ and $2$~mbar.
The two gas volumes will be separated by $50$~$\mu$m Mylar, covered with $1$~$\mu$m $Al$.
To limit the variations of the drift field, we require the pressure difference between the two gas
volumes to be less than $0.2$~mbar. 

Oxygen contamination and water vapor should be kept less than $50$~ppm, 
to minimize the electron recombination 
in the drift volume. The Nitrogen contamination should be kept less than $0.5\%$.

The elements of the gas system that operate above $1$~bar should be kept in a separate gas room,
elevated approximately $7$~m above the detector. They will be connected to the detector with
gas lines of about $50$~m length.

The parameters and requirements of the gas system are summarized in Table \ref{tab:param}.

\begin{table}[hp]
\begin{ruledtabular}
\begin{tabular}{lcc}
\textrm{item}&
\textrm{requirement}&
\textrm{comment}\\
\colrule
total $Xe/CO_2$  gas volume & $300$ l &\\
total $CO_2$ gas volume & $1300$ l &\\
$Xe/CO_2$  gas flow & $100$ l/h &\\
$CO_2$ gas flow & $55$ l/h &\\
Operating overpressure & $0.5-2$~mbar &\\
Pressure difference b/n two volumes & $<0.2$~mbar &\\
Oxygen contamination & $<50$~ppm &\\
water vapor & $<50$~ppm &\\
Nitrogen contamination & $<0.5\%$ &\\
\end{tabular}
\end{ruledtabular}
\caption{
General parameters and requirements for the GlueX TRD gas system.
\label{tab:param}
}
\end{table}


\bibliography{trd_gas.bib}% Produces the bibliography via BibTeX.

\end{document}
%
% ****** End of file apssamp.tex ******
