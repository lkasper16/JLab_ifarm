% ****** Start of file apssamp.tex ******
%
%   This file is part of the APS files in the REVTeX 4.1 distribution.
%   Version 4.1r of REVTeX, August 2010
%
%   Copyright (c) 2009, 2010 The American Physical Society.
%
%   See the REVTeX 4 README file for restrictions and more information.
%
% TeX'ing this file requires that you have AMS-LaTeX 2.0 installed
% as well as the rest of the prerequisites for REVTeX 4.1
%
% See the REVTeX 4 README file
% It also requires running BibTeX. The commands are as follows:
%
%  1)  latex apssamp.tex
%  2)  bibtex apssamp
%  3)  latex apssamp.tex
%  4)  latex apssamp.tex
%
\documentclass[%
%%%%%ok reprint,
%superscriptaddress,
%groupedaddress,
%unsortedaddress,
%runinaddress,
%frontmatterverbose, 
preprint,
%showpacs,preprintnumbers,
nofootinbib,
%nobibnotes,
%bibnotes,
 amsmath,amssymb,
 aps,
%pra,
%prb,
%rmp,
%prstab,
%prstper,
floatfix,
]{revtex4-1}

\usepackage{graphicx}% Include figure files
\usepackage{setspace}
\usepackage{dcolumn}% Align table columns on decimal point
\usepackage{bm}% bold math
%\usepackage[caption=false]{subfig}% bold math
\usepackage{subcaption}% bold math
%\usepackage{hyperref}% add hypertext capabilities
%\usepackage[mathlines]{lineno}% Enable numbering of text and display math
%\linenumbers\relax % Commence numbering lines

%\usepackage[showframe,%Uncomment any one of the following lines to test 
%%scale=0.7, marginratio={1:1, 2:3}, ignoreall,% default settings
%%text={7in,10in},centering,
%%margin=1.5in,
%%total={6.5in,8.75in}, top=1.2in, left=0.9in, includefoot,
%%height=10in,a5paper,hmargin={3cm,0.8in},
%]{geometry}

\begin{document}

\preprint{Notes for discussion}

\title{How to reconcile the results for the J/$\psi p$ scattering length obtained from 
the total and differential photoproduction cross sections}% Force line breaks with \\
%\thanks{Thanks to everybody who contributed to this work}%
\author{Lubomir Pentchev}
\altaffiliation{Thomas Jefferson National Accelerator Facility, Newport News, Virginia 23606, USA}
\author{Igor Strakovsky}
\altaffiliation{Institute for Nuclear Studies, Department of Physics,
The George Washington University, Washington, DC 20052, USA}
\author{Alexander Titov}
\altaffiliation{Bogoliubov Laboratory of Theoretical Physics, JINR, Dubna 141980, Russia}
%
%\author{Ann Author}
% \altaffiliation[Also at ]{Physics Department, XYZ University.}%Lines break automatically or can be forced with \\
%\author{Second Author}%
% \email{Second.Author@institution.edu}
%\affiliation{%
% Authors' institution and/or address\\
% This line break forced with \textbackslash\textbackslash
%}%

%\collaboration{GlueX Collaboration}%\noaffiliation


\date{\today}% It is always \today, today,
             %  but any date may be explicitly specified

\begin{abstract}
In Ref. \cite{Vanderhaeghen_jpsi}
a value for the $J/\psi p$ scattering length of
$a_{\psi p} = 0.046 \pm 0.005$~fm is extracted
mainly constrained by the $J/\psi $ differential
photoproduction cross section $d\sigma/dt$ at threshold.
In another approach \cite{Strakovsky_jpsi} using the $J/\psi $ total cross section near threshold,
a value of $a_{\psi p} = 0.00308 \pm 0.00055(stat.) \pm 0.00045(syst.)$~fm is estimated.
Both methods use the Vector Meson Dominance approximation to relate the $\gamma p \to J/\psi p$ to the
$J/\psi p \to J/\psi p$ cross section and extract the scattering length.
In this note we study this, more than an order of magnitude, discrepancy,
present a reconciliation of the problem, 
and discuss possible implications.
\end{abstract}

%\pacs{Valid PACS appear here}% PACS, the Physics and Astronomy
%                             % Classification Scheme.
%%\keywords{Suggested keywords}%Use showkeys class option if keyword
%                              %display desired
\maketitle

%\tableofcontents

\clearpage
\mbox{~}
%\section{Formulation of the problem}

The $J/\psi $ photoproduction reaction, $\gamma p \to J/\psi p$, can give access to the 
$J/\psi p$ elastic scattering by using the Vector Meson Dominance (VMD) model.
In Ref. \cite{Vanderhaeghen_jpsi} a method to extract the real and imaginary parts of
the $J/\psi p$ forward amplitude $T^{\psi p}$ is presented.
It is based on fits of the elastic $\gamma p \to J/\psi p$ and inelastic
$\gamma p \to c\bar{c} X$ total cross sections 
and the forward differential elastic cross section $\frac{d\sigma^{\gamma p}}{dt}\vert _{t=0}$.
The imaginary part is extracted using the optical theorem and 
parametrizations of the total cross sections.  
Then the real part is obtained using dispersion relation with one subtraction.
The subtraction constant is defined by the real part of the amplitude at threshold.
Using VMD, the latter is constrained by the forward differential cross-section near threshold:
\begin{equation}
\frac{d\sigma^{\gamma p}}{dt}\Big| ^{t=0} = 
\frac{\alpha \pi}{\gamma ^2}\frac{q_{\psi p}^2}{k_{\gamma p}^2}
\frac{d\sigma^{\psi p}}{dt}\Big| ^{t=0} =
\frac{\alpha \pi}{\gamma ^2}\frac{1}{64\pi s k_{\gamma p}^2}|T^{\psi p}|^2,
\end{equation}
as the imaginary part vanishes there. 
Here $q_{\psi p}$ and $k_{\gamma p}$ are the COM momenta
of the final and initial particles, and $\gamma _\psi$ is the photon-$J/\psi $ coupling
obtained from the $J/\psi \to e^+e^-$ decay.  
On the other hand, the forward amplitude at threshold 
is related to the $J/\psi $ scattering length, $a_{\psi p}$:
\begin{equation} 
T^{\psi p}\vert_{thresh} = 8\pi(m+M)a_{\psi p},
\end{equation} 
where $m+M$, the sum of proton and $J/\psi $ masses, represents the total energy of the system 
$W=\sqrt{s}$ at threshold.
Therefore in Ref. \cite{Vanderhaeghen_jpsi} the scattering length is obtained
practically from the equation:
\begin{equation}
\frac{d\sigma^{\gamma p}}{dt}\Big| ^{t=0}_{thresh} = 
\frac{\alpha \pi}{\gamma ^2}\frac{\pi }{k_{\gamma p}^2}a_{\psi p}^2.
\label{eq:method1}
\end{equation}
The left-hand side is not a quantity that can be  measured directly, as it requires 
both, extrapolation in energy to the threshold and extrapolation in $t$
from the physical region $t_{min}-t_{max}$ to the nonphysical point $t=0$.
The differential cross-section data closest to the threshold 
that is used in Ref. \cite{Vanderhaeghen_jpsi} is from the SLAC \cite{SLAC}
measurements above $13$~GeV, while the energy threshold is at $8.2$~GeV.
The best fit results in a value for the scattering  
length of $a_{\psi p} = 0.046 \pm 0.005$~fm.


In Ref. \cite{Strakovsky_jpsi} another approach is used to estimate the 
$J/\psi p$ scattering length. It relies on the recent measurement
of the total $J/\psi $ photoproduction cross section near threshold 
from the GlueX collaboration \cite{prl_gluex}.
Within the VMD model the total $\gamma p \to J/\psi p$ cross section is
related to the total $J/\psi p \to J/\psi p$ cross section 
and, at threshold, to the scattering length by \cite{Titov}: 
\begin{equation}
\sigma^{\gamma p}| _{thresh} = 
\frac{\alpha \pi}{\gamma ^2}\frac{q_{\psi p}}{k_{\gamma p}}
\sigma^{\psi p}| _{thresh} =
\frac{\alpha \pi}{\gamma ^2}\frac{q_{\psi p}}{k_{\gamma p}}
4\pi a_{\psi p}^2,
\label{eq:method2}
\end{equation}
When approaching the threshold, $q_{\psi p} \rightarrow 0$.
Therefore, with this method, the derivative of the cross section as function
of $q_{\psi p}$, for $q_{\psi p} \rightarrow 0$, is estimated from the data 
and then related to the scattering length by the above equation.
The result is $a_{\psi p} = 0.00308 \pm 0.00055(stat.) \pm 0.00045(syst.)$~fm,
which is more than an order of magnitude smaller than 
the scattering length obtained with the first method \cite{Vanderhaeghen_jpsi}.
Such a difference can not be explain only by the uncertainties
of the calculations. 
The extrapolations, especially in the first method as discussed above w.r.t. to Eq. (\ref{eq:method1}), 
result in significant uncertainties, however still not enough to explain this big discrepancy.

As a first step to understand the difference between the two methods,
we use the obvious relation between the 
total and differential cross section:
\begin{equation}
\sigma^{\gamma p}=\int _{t_{min}}^{t_{max}} \frac{d\sigma^{\gamma p}}{dt} dt 
\end{equation}
When approaching the threshold, $t_{min} \rightarrow t_{max}$, and 
$\Delta t = t_{min} - t_{max} \rightarrow 4q_{\psi p}k_{\gamma p}$:
\begin{equation}
\sigma^{\gamma p}|_{thresh}=4q_{\psi p}k_{\gamma p}\frac{d\sigma^{\gamma p}}{dt} \Big|_{t=t_{thresh}} 
\label{eq:totdiff}
\end{equation}
which relates the total and differential cross section at threshold; 
here $t_{thresh}=t_{min}=t_{max}$.
The same result can be obtained from the relation \cite{Titov}:
\begin{equation}
\frac{d\sigma^{\gamma p}}{dt}\Big| _{thresh} = 
\frac{\alpha \pi}{\gamma ^2}\frac{\pi }{k_{\gamma p}^2}a_{\psi p}^2.
\end{equation}
and Eq.(\ref{eq:method2}).

In the next step, 
based on Eq.(\ref{eq:totdiff}), we will test the consistency 
of the GlueX data in extracting the scattering length.
This would serve also as a validation of the extrapolation in energy used 
in the second method \cite{Strakovsky_jpsi}.
In Ref.\cite{Brodsky} the asymptotic behavior of the $J/\psi $ photoproduction near threshold
is studied using the dimensional scaling.
Due to the OZI rule, the $J/\psi -p$ interaction is mediated predominantly by gluons.
In this approach the cross section depends on the number of partons, $n$ (hard gluons, resp. quarks),
participating in the reaction:
\begin{equation}
\frac{d\sigma^{\gamma p}}{dt} \sim 
(1-x)^{3-n}F^2(t),
\label{eq:brodsky}
\end{equation}
where $x=(2mM+M^2)/(s-m^2)$ and $F(t)$ is the proton gluonic form factor.
In Fig.\ref{fig:xsec_fit} the GlueX \cite{prl_gluex} 
and SLAC \cite{SLAC} total cross sections as function of $q_{\psi p}$,
are fitted with a sum (incoherent)
of two- and three-gluon exchange curves with two parameters being the amplitudes 
of the corresponding contributions.
\begin{figure}[h]
\includegraphics[width=0.80\textwidth]{./fig/Jpsi_sl_xsec_fit.pdf}
  \caption{
Total $J/\psi $ photoproduction cross sections from GlueX \cite{prl_gluex}
and SLAC \cite{SLAC} fitted with two- and three-gluon exchange curves \cite{Brodsky}
and odd-power polynomial as in \cite{Strakovsky_jpsi}, as function of the CM momentum of 
the final state particles.
The SLAC total cross sections are obtained from 
the differential ones using the procedure in \cite{prl_gluex}.
}
  \label{fig:xsec_fit}
\end{figure}
We find, as in \cite{prl_gluex}, that the three-gluon exchange dominates in the GlueX energy region,
which  is expected as near the threshold all partons should participate in the reaction.
This finding results in significant simplifications of the analysis as 
according to Eq.(\ref{eq:brodsky}) for $n=3$ 
{\bf the differential cross section does not depend on the energy}.
Besides other important consequences that will be discussed at the end,
this means that we can use  measurements of the $t$- dependence 
at energies away from the threshold to predict the cross section at the threshold.
A model of such a cross-section vs beam energy and $t$ 
is illustrated in Fig.\ref{fig:xsec_model}.
\begin{figure}
%  \begin{subfigure}[b]{0.45\textwidth}
    \includegraphics[width=0.6\textwidth]{./fig/Jpsi_sl_xsec_model.pdf}
    \caption{Model of the $J/\psi $ photoproduction cross section 
(in arbitrary units), 
with no energy dependence near threshold,  as function of energy and $t$.}
    \label{fig:xsec_model}
\end{figure}

The GlueX collaboration reported also the differential cross section
as function of $t$ for an average energy of $10.7$~GeV \cite{prl_gluex}. 
Fit with an exponential function,
%\begin{equation}
$d\sigma^{\gamma p}/dt = Ae^{b(t-t_{min})}$ 
%\frac{d\sigma^{\gamma p}}{dt} = Ae^{b(t-t_{min})} 
%\end{equation}
results in a slope of $b=1.67 \pm 0.35$~GeV$^2$, and $A=1.48 \pm 0.26$.
Using $t_{min}=0.55$~GeV$^2$ for this energy and $t_{thresh} = 2.23$~GeV$^2$ 
we estimate for the right-hand side of Eq.(\ref{eq:totdiff}) a $q$-slope of :
\begin{equation}
4q_{\psi p}k_{\gamma p}\frac{d\sigma^{\gamma p}}{dt}\Big |_{t=t_{thresh}} =
q_{\psi p} (0.69\pm0.31)\;nb/GeV
\label{eq:rhs}
\end{equation}
For the left-hand side of Eq.(\ref{eq:totdiff}), 
it is found in Ref.\cite{Strakovsky_jpsi} using an odd-power 
polynomial fit to the GlueX total cross section,
a slope of $0.46 \pm 0.16$~nb/GeV. 
If we calculate the derivative at $q=0$ of the three-gluon exchange curve
that was used to fit the GlueX total cross section (See Fig.\ref{fig:xsec_zoom})
we get a value of $0.64\pm0.09$~nb/GeV. 
Both values are in good agreement with the right-hand side given by Eq.(\ref{eq:rhs}).
%  \end{subfigure}
%
%  \begin{subfigure}[b]{0.45\textwidth}
\begin{figure}
    \includegraphics[width=0.6\textwidth]{./fig/Jpsi_sl_xsec_fit_zoomed.pdf}
    \caption{Fig.\ref{fig:xsec_fit} zoomed near threshold.}
    \label{fig:xsec_zoom}
%  \end{subfigure}
\end{figure}

Thus, we have verified, based on Eq.(\ref{eq:totdiff}),
that when using the total or the differential cross sections
the results for the scattering length are very similar
and this itself is not an explanation for the discrepancy between the two methods.
However, there is an important difference between 
Eq.(\ref{eq:method1}) used in the first method \cite{Vanderhaeghen_jpsi} and
Eq.(\ref{eq:totdiff}) that represents the second method \cite{Strakovsky_jpsi}. 
In the former case, the differential cross-section $d\sigma^{\gamma p}/dt$
is taken at $t=0$, while in the latter case it is at $t=t_{thresh}$.
This is a result of the fact that when the total cross section approaches 
the energy threshold ($q_{\psi p}=0$) $t$ approaches $t_{thresh}$. 
On the other hand when using the differential cross section we can
not only extrapolate in energy to the threshold, but also in $t$ to $t=0$.
While for the $J/\psi p \to J/\psi p$ elastic reaction,
the forward scattering is equivalent to $t=0$, this is not the case for the photoproduction.

We can estimate the difference between the two results using again the fit of the GlueX differential cross
section. We obtain $d\sigma^{\gamma p}/dt (t=0) = 3.7 \pm 1.1$~nb/GeV$^2$, 
while $d\sigma^{\gamma p}/dt (t=t_{thresh}) = 0.09 \pm 0.4$~nb/GeV$^2$.
In addition the SLAC cross-section measurements used in the first method
are done away from the threshold and the best fit 
extrapolates them to $d\sigma^{\gamma p}/dt (t=0) \approx 10$~nb/GeV$^2$ at the threshold
(see Fig.3 in Ref. \cite{Vanderhaeghen_jpsi}).
Thus, the differential cross-section method uses a value for $d\sigma^{\gamma p}/dt$ 
that is more than two orders of magnitude bigger than the value from Eq.\ref{eq:totdiff}
that corresponds to the total cross section at threshold. This explains the more than an order
of magnitude discrepancy between the two methods for the scattering length that enters quadratically.
 
The independence of the $J/\psi $ photoproduction cross section on the energy
discussed above may have other important applications when estimating quantities at the threshold, 
like the trace anomaly of the energy-momentum tensor ...





%\section{Solution and discussions}

%\section{Other implications}

\bibliography{jpsi_sl.bib}% Produces the bibliography via BibTeX.

\end{document}
%
% ****** End of file apssamp.tex ******
